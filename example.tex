\documentclass[12pt,a4paper]{article}

% --------------------------------------------------------
% PACKAGES
% --------------------------------------------------------
\usepackage[utf8]{inputenc}
\usepackage[T1]{fontenc}
\usepackage{geometry}
\usepackage{graphicx}
\usepackage{hyperref}
\usepackage{amsmath, amssymb}
\usepackage{float}
\usepackage{array}
\usepackage{booktabs}
\usepackage{setspace}
\usepackage{xcolor}

\geometry{a4paper, margin=1in}
\graphicspath{{./images/}}

\title{\textbf{Automata Theory and Computability Assignment}}
\author{Vivek}
\date{\today}

\begin{document}
\maketitle
\onehalfspacing

% --------------------------------------------------------
% 1. INTRODUCTION
% --------------------------------------------------------
\section{Introduction}

This report presents the design and analysis of several computational models in the theory of computation, including Finite Automata (FA), Pushdown Automata (PDA), and Turing Machines (TM).  
For each language, appropriate computational models are constructed and verified using JFLAP.  
The report is organized so that \textbf{each question has its methodology, results, and discussion grouped under Sections 3, 4, and 5 respectively}.

% --------------------------------------------------------
% 2. PROBLEM STATEMENTS
% --------------------------------------------------------
\section{Problem Statements}

\begin{verbatim}
1.  Design a NPDA for L = { a^n b^n | n >= 1 }
2.  Construct DFA for L = { a b^n a^m : n >= 2, m >= 3 }
3.  DFA for strings ending with '0011', Σ = {0,1}
4.  NPDA for L = { a^n b^m c^(n+m) }
5.  Minimal DFA where 'a' is never followed by 'bb'
12. NFA for {ab, abc}*
13. DFA for strings ending with 'abb'
14. DFA for strings ending with 'abba'
15. DFA/NFA for strings containing “the”
16. DFA/NFA for strings ending with “ing”
29. TM accepting palindromes over {a,b}
30. TM accepting { w w^R }
24. TM accepting { 0^n 1^n }
25. TM accepting { 0^n 1^n 2^n }
26. TM for strings containing substring 001
\end{verbatim}

% --------------------------------------------------------
% 3. METHODOLOGY
% --------------------------------------------------------
\section{Methodology}

This section details the construction approach for each question.

% --------------------------------------------------------
% Question 1 Methodology
% --------------------------------------------------------
\subsection{Methodology for Question 1: NPDA for $a^n b^n$}

\begin{itemize}
    \item Identify that equal matching requires a stack → use PDA.
    \item Push all input \texttt{a}'s.
    \item On first \texttt{b}, transition to popping mode.
    \item Accept only when stack returns to bottom marker.
\end{itemize}

\begin{figure}[H]
    \centering
    \includegraphics[width=0.7\textwidth]{q1_Screenshot_2025_11_28_203141.png}
    \caption{NPDA for Question 1 (Part 1)}
\end{figure}

\begin{figure}[H]
    \centering
    \includegraphics[width=0.7\textwidth]{q1_Screenshot_2025_11_28_203152.png}
    \caption{NPDA for Question 1 (Part 2)}
\end{figure}

% --------------------------------------------------------
% Question 2 Methodology
% --------------------------------------------------------
\subsection{Methodology for Question 2: DFA for $ab^n a^m$}

\begin{itemize}
    \item Enforce structure: 1 a → $\ge 2$ b's → $\ge 3$ a's.
    \item Dedicated states ensure minimal required counts.
    \item Loop transitions handle additional b's and a's.
\end{itemize}

\begin{figure}[H]
    \centering
    \includegraphics[width=0.7\textwidth]{q2_image.png}
    \caption{DFA for Question 2}
\end{figure}

% You continue this same pattern for Questions 3–26
% Each question gets its own subsection under Methodology
% --------------------------------------------------------

% --------------------------------------------------------
% 4. RESULTS
% --------------------------------------------------------
\section{Results}

This section presents acceptance tables and correctness verification for each question.

% --------------------------------------------------------
% Question 1 Results
% --------------------------------------------------------
\subsection{Results for Question 1}

\begin{center}
\begin{tabular}{|c|c|c|}
\hline
\textbf{String} & \textbf{Result} & \textbf{Reason} \\ \hline
ab & Accept & 1 a, 1 b \\ \hline
aabb & Accept & Equal count \\ \hline
a & Reject & Missing b \\ \hline
abb & Reject & Extra b \\ \hline
\end{tabular}
\end{center}

% --------------------------------------------------------
% Question 2 Results
% --------------------------------------------------------
\subsection{Results for Question 2}

\begin{center}
\begin{tabular}{|c|c|c|}
\hline
String & Result & Reason \\ \hline
abbaaa & Accept & 2 b, 3 a \\ \hline
abbbaaaa & Accept & n>2, m>3 \\ \hline
abaaa & Reject & Only 1 b \\ \hline
\end{tabular}
\end{center}

% Continue for all Questions …

% --------------------------------------------------------
% 5. DISCUSSION
% --------------------------------------------------------
\section{Discussion}

This section contains per-question interpretation of correctness and conceptual insights.

\subsection{Discussion for Question 1}

The NPDA successfully models the matching requirement by using a stack to count \texttt{a}'s.  
This highlights the limitations of finite automata and the necessity of PDAs for context-free languages.

\subsection{Discussion for Question 2}

The DFA structure ensures strict enforcement of symbol counts through state progression.  
This shows how DFAs encode counting constraints through fixed sequences of states.

% Continue similarly for Q3–Q26 …

% --------------------------------------------------------
% 6. COMPARATIVE ANALYSIS
% --------------------------------------------------------
\section{Comparative Analysis}

\begin{center}
\begin{tabular}{|c|c|c|c|}
\hline
Model & Memory & Language Class & Example \\ \hline
DFA & None & Regular & Ending with 0011 \\ \hline
PDA & Stack & Context-Free & $a^n b^n$ \\ \hline
TM & Infinite Tape & RE/Type-0 & $0^n1^n2^n$ \\ \hline
\end{tabular}
\end{center}

% --------------------------------------------------------
% 7. CONCLUSION
% --------------------------------------------------------
\section{Conclusion}

This assignment demonstrated practical construction of automata across the Chomsky hierarchy.  
Using JFLAP, each automaton was verified using both valid and invalid strings, confirming correctness and deepening understanding of language recognition models.

% --------------------------------------------------------
% 8. REFERENCES
% --------------------------------------------------------
\section{References}

\begin{thebibliography}{9}
\bibitem{hopcroft}
Hopcroft, Motwani, Ullman. \textit{Introduction to Automata Theory, Languages, and Computation}.

\bibitem{sipser}
Sipser. \textit{Introduction to the Theory of Computation}.

\bibitem{linz}
Linz. \textit{Formal Languages and Automata}.
\end{thebibliography}

\end{document}
